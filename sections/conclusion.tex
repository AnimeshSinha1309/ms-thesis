In this dissertation, we have shown that Quantum-inspired Machine-Learning methods can be naturally applied in various parts of the Quantum processing pipeline. We present two novel results: qRoute - a reinforcement-learning based quantum circuit routing method, and qLEET - a variational quantum circuit visualization and evaluation framework.

In qRoute, we have shown state-of-the-art performance on a large variety of quantum circuit compilation tasks. This demonstrates that reinforcement learning settings are capable of learning algorithms more performant than manually engineered heuristics. The routing task is one step of the long hardware-efficient algorithm development and compilation pipeline. The solution for the routing task is also a proposed solution to the general problem of decision over subsets.

In qLEET, we have shown framework consisting of both optimization-inspired and quantum-inspired methods to evaluate the learning capabilities of variational quantum circuits. These visualized metrics include loss landscape, training trajectories, expressibility, entanglement capability and parameter histogram of these circuits. The framework presented is provided with code compatible with Cirq, Qiskit and PyTket quantum circuits and works with any choice of loss function, optimizer, and noise model. Our method helps guide better ansatz design and parametrization for solving these variational problems.
The software package for qRoute has been made publically available on \url{https://github.com/AnimeshSinha1309/qroute-router/}.

The routing process is just one part of the compilation process, where we convert a circuit that doesn't respect the constraints of the hardware to one that does. Converting a unitary directly to a hardware-constraint-respecting circuit, and optimizing noise under a specific noise model instead of minimizing depth are both goals that we need to design algorithms for, and qRoute seems like a viable candidate as a part of the pipeline for these processes. There is also scope for adaptation of the qRoute algorithm for these objectives. Similarly, qLEET can be extended with more visualizations and analysis of the optimization landscape and quantum resource utilization. qLEET has been explored on a small class of algorithms like QAOA and VQE, visualization of other algorithms and variation algorithm design and hyperparameter tuning using qLEET has been left as future scope of the project. The software package for qLEET has been made publically available on \url{https://github.com/QLemma/qleet}.


On the entirety, we believe that the contributions of this dissertation will help better the time-efficiency and noise-immunity of some algorithms on noisy near-term intermediate-scale quantum computers and will serve as some inspiration to pipelines for other tasks in near-term quantum algorithm design.
